\chapter{Implementação do sistema}

\section{Implementação e adaptação do \textit{backend}}

\subsection{Servidor de Autorização e Autenticação}
O servidor de autenticação e autorização disponibiliza uma \gls{API} para criação de novos utilizadores e solicitação de \textit{tokens} de acesso relativamente a várias aplicações. Este serviço utiliza a plataforma \textit{Spring}, mais concretamente a \textit{Spring Security} \cite{spring-framework}. \par 
No nosso caso vamos apenas ter uma aplicação cliente configurada para posteriormente gerir os \textit{tokens} de acesso para esta aplicação. As credenciais das aplicações cliente vão estar guardadas num base de dados relacional e os utilizadores finais numa base de dados não relacional. Esta opção está relacionada com a segurança de dados referida anteriormente em \ref{cap6:protecaodados}. 
Relativamente à base de dados não relacional para os utilizadores finais são guardados numa coleção do \textit{MongoDB}\cite{mongodb}, a coleção é denominada por \textit{endUser}. Este coleção é então composta por vários documentos, em que cada documento corresponde a um utilizador final. Os documentos são formatados como na Figura \ref{f:endUserCode}.

\begin{figure}[H]
\inputminted[fontsize=\scriptsize]{json}{code/endUser.json}
\caption[Formato do documento de um utilizador final]{Formato do documento de um utilizador final}
\label{f:endUserCode}
\end{figure}

Para a base de dados relacional a escolha podia por exemplo ser efetuada entre o  \textit{MySQL} \cite{mysql} e  \textit{PostgreSQL} \cite{postgresql}, neste caso foi utilizado o  \textit{PostgreSQL}, o modelo de dados relacional está representado na Figura \ref{f:auth-diagram}. \par 
Na tabela  \textit{oauth\_client\_details} estão guardadas todas as informações relativas à aplicação como por exemplo a sua identificação, a chave secreta, os tipos de autorização, etc. Relativamente à tabela  \textit{oauth\_access\_token} estão guardados os  \textit{tokens} de acesso por utilizador, relativamente a uma aplicação. Cada  \textit{token} de acesso tem relacionado a ele  um  \textit{refresh token} que está guardado numa tabela diferente, na tabela  \textit{oauth\_refresh\_token}.

 \begin{figure}[H]
  \centering
  \includegraphics[width=0.8\textwidth]{imgs/auth-diagram.png}
  \caption[Diagrama do modelo de dados para guardar as credenciais dos clientes e os  \textit{tokens} de acesso]{Diagrama do modelo de dados para guardar as credenciais dos clientes e os  \textit{tokens} de acesso}
  
  \label{f:auth-diagram}
\end{figure}

\subsubsection{API REST}
\label{l:restapiAUTH}
Na tabela \ref{t:apirest-auth} temos os métodos disponíveis do servidor de autorização e autenticação.
\begin{table}[H]

\centering
\begin{tabularx}{1\textwidth}{|p{0.3cm} p{14.4cm}|}
\multicolumn{2}{l}{\textbf{domain:port/users}}  \\ \hline 
 & Serviço POST, permite a criação de novos utilizadores. \\
 & Recebe um JSON composto por dois campos um  \textit{username} e  \textit{password}. \\ \hline
\multicolumn{2}{l}{\textbf{domain:port/oauth/token}} \\ \hline
 & Serviço POST, permite o pedido de um novo  \textit{token} de acesso. \\
 & Recebe por parâmetros o  \textit{username},  \textit{password} e o  \textit{grant\_type}. \\ \hline
\end{tabularx}
\caption{Tabela com a API REST do servidor de autorização e autenticação}
\label{t:apirest-auth}
\end{table}

\subsection{Servidor de recursos ( \textit{dataPoint} REST API) }
O servidor de recursos disponibiliza uma \gls{REST} \gls{API} para criação, consulta e eliminação de  \textit{dataPoints}. Estes  \textit{dataPoints} estão a ser guardados numa base de dados  \textit{MongoDB} numa coleção com o nome  \textit{dataPoint}. A maior vantagem de utilizar uma base de dados não relacional é que os esquemas de dados podem ser estendidos de forma mais flexível, não sendo necessário efetuar nenhuma alteração ao esquema da BD. 
\par 
Como foi referido na fase exploratória do \gls{OMH} \ref{cap4:exp-omh} estes  \textit{dataPoints} são compostos por um cabeçalho e por um corpo. Este corpo é formatado pelo esquema de dados ( \textit{data schema}) associado e definido no cabeçalho do  \textit{dataPoint}. Como foi referido também na fase exploratória foi desenvolvido um validador que verifica se o corpo do  \textit{dataPoint} está de acordo com o  \textit{data schema} definido no cabeçalho do mesmo. \par
Estavam então em falta alguns esquemas de dados para conseguirmos abranger todos os tipos dados necessários, entre eles: \gls{ECG}, dados demográficos e acelerómetro. Estes  \textit{data schemas} foram criados aproveitando as estruturas da definição da \gls{API} do \gls{FHIR} e vou mostrar de seguida como exemplo o esquema de dados criado para definir o formato de dados para as entradas para o \gls{ECG}. \newpage
\subsubsection{ \textit{data schema} ECG}
\begin{figure}[H]
\inputminted[fontsize=\scriptsize]{json}{code/ecg.json}
\caption[Novo esquema de dados relativo ao ECG]{Novo esquema de dados relativo ao ECG}
\label{f:ecgjsonschema}
\end{figure}

\subsubsection{API REST}
Na tabela \ref{t:apirest-data} temos os métodos disponíveis do servidor de recursos.
\label{l:restapiRESOURCES}

\begin{table}[H]
\centering
\begin{tabularx}{1\textwidth}{|p{0.3cm} p{14.4cm}|}
\multicolumn{2}{l}{\textbf{domain:port/v\{apiVersion\}/dataPoints}}  \\ \hline 
 & Serviço POST, para criação de novos  \textit{dataPoints}. \\
 & No  \textit{header} recebe o  \textit{access\_token} para o servidor de autorização lhe dar acesso ao pedido efetuado. Todos os pedidos efetuados a esta \gls{API} tem que fornecer o \textit{token} de acesso.  \\
 & Recebe um documento JSON constituído por um cabeçalho e um corpo. \\ \hline
 
 
\multicolumn{2}{l}{\textbf{domain:port/v\{apiVersion\}/dataPoints/id}} \\ \hline
 & Serviço DELETE, permite a eliminação de um  \textit{dataPoint}. \\
 & O  \textit{id} do  \textit{dataPoint} tem que ser referido no url. \\ \hline
 
 
 \multicolumn{2}{l}{\textbf{domain:port/v\{apiVersion\}/dataPoints/id}} \\ \hline
 & Serviço GET, permite a consulta de um  \textit{dataPoint}. \\
 & O  \textit{id} do  \textit{dataPoint} tem que ser referido no url. \\ \hline
 
 
 \multicolumn{2}{l}{\textbf{domain:port/v\{apiVersion\}/dataPoints/caregiver}} \\ \hline
 & Serviço GET, permite a consulta dos vários  \textit{dataPoints} de um determinado tipo de um participante em específico. \\
 & Recebe por parâmetros o  \textit{schema\_namespace},  \textit{schema\_name},  \textit{schema\_version},  \textit{user\_id} e o  \textit{CAREGIVER\_KEY}. \\ \hline
 
 
 \multicolumn{2}{l}{\textbf{domain:port/v\{apiVersion\}/dataPoints}} \\ \hline
 & Serviço GET, permite a consulta dos vários dataPoints de um determinado esquema de dados. \\
 & Recebe por parâmetros o  \textit{schema\_namespace},  \textit{schema\_name} e o  \textit{schema\_version}. \\ \hline
\end{tabularx}

\caption{Tabela com a API REST do servidor de recursos}
\label{t:apirest-data}
\end{table}


Todos os esquemas de dados disponíveis estão no repositório associado à adaptação deste \textit{backend}\footnote{ Diretório com os Esquemas de dados disponíveis(composto também pelos novos esquemas de dados criados) - https://github.com/luistduarte/omh-dsu-ri/tree/master/resource-server/src/main/resources/schema/omh} e estes são os tipos de dados que são utilizados para validar os  \textit{dataPoints} inseridos, ou seja, estes são todos aqueles que podem ser inseridos na coleção do  \textit{MongoDB}.

\subsection{Comunicação com módulos externos}
Para a possível integração com módulos externos foi utilizada a plataforma do Vert.x. No processo de inserção de novos \textit{dataPoints}, depois de validados eles são reencaminhados para um \textit{endpoint} \gls{REST}. É enviado um documento \gls{JSON} composto pelo esquema de dados utilizado para validar o documento, como por exemplo \textit{heart-rate} e um corpo do \textit{dataPoint} inserido. Este documento é então inserido no \textit{EventBus} do Vert.x. Neste momento os módulos que subscreveram o esquema de dados \textit{heart-rate} podem verificar e analisar da maneira que quiserem os dados inseridos. Temos na figura \ref{f:toExtSeqDiagram} um diagrama de sequência com este fluxo. \par
Isto pode ser importante para por exemplo ser desenvolvido um módulo que tenha a capacidade de verificar se um utilizador que está a fazer a colheita dos dados está a sofrer de arritmia cardíaca.
O projeto é então composto por dois \textit{verticles} em que um dos \textit{verticles} cria um \textit{endPoint} \gls{REST} para receber os dados e os publicar no \textit{Eventbus} assim que os recebe.

\begin{figure}[H]
  \centering
  \includegraphics[width=1\textwidth]{imgs/toExtSeqDiagram.png}
  \caption[Diagrama de sequência com a comunicação com módulos externos]{Diagrama de sequência com a comunicação com módulos externos}
  
  \label{f:toExtSeqDiagram}
\end{figure}

\subsection{Como usar o \textit{backend} desenvolvido num projeto}
O sistema foi desenvolvido para ser fácil de utilizar em outros projetos. Tem a capacidade de estar a ser utilizado por diferentes aplicações \textit{mHealth} ao mesmo tempo. Tudo está pronto a utilizar e pode ficar ainda mais completo, pois podem por exemplo ser adicionados novos esquemas de dados. \par 
Relativamente ao servidor de autorização/autenticação e ao servidor de recursos o código está no github \footnote{https://github.com/luistduarte/omh-dsu-ri} e é um \textit{fork} do projeto desenvolvido pela \gls{OMH}, está documentado como pode ser instalado e executado\footnote{https://github.com/luistduarte/omh-dsu-ri/blob/master/README.md\#option-2-building-from-source-and-running-nativel}.
O código relativo à extensibilidade com módulos externos também está no github\footnote{https://github.com/luistduarte/rest.pub.sub}, este projeto é composto apenas por dois verticles para os correr pode ser também por terminal e basta correr os comandos:
\begin{itemize}
    \item vertx run SimpleREST.java -cluster
    \item vertx run Receiver.java -cluster
\end{itemize}
O Receiver.java pode ser substituído ou adicionado por um outro módulo desenvolvido para receber outro tipo de esquemas de dados e os analisar.

\section{Implementação da aplicação móvel}
\subsection{Versões alvo e dependências}
Esta aplicação móvel foi desenvolvida com o intuito de conseguir abranger grande maioria dos utilizadores, suporta da versão 16 à versão 25, ou seja,  desde o \textit{Android} 4.1 (\textit{Jelly Bean}) até ao \textit{Android} 7.1. De acordo com as informações disponíveis na plataforma de desenvolvimento do \textit{Android} atinge mais de 98\% dos utilizadores \cite{android-versions}.

O projeto tem um ficheiro por defeito designado por "build.gradle", que é o local onde é efetuada a configuração relativa às versões alvo e às dependências existentes, excertos deste ficheiro de configuração podem ser visualizados nas tabelas \ref{t:defaultconfigversions} e \ref{t:defaultconfigdependency} .


\begin{table}[H]
\centering
\begin{tabular}{|llll|}
\hline
 & \multicolumn{2}{l}{defaultConfig \{}                                     &  \\
 &    & applicationId {\color[HTML]{009901} ''pt.ua.ieeta.healthintegration''}                       &  \\
 &    & minSdkVersion {\color[HTML]{0767D2}16}                                                    &  \\
 &    & targetSdkVersion {\color[HTML]{0767D2}25}                                                 &  \\
 &    & versionCode {\color[HTML]{0767D2}1}                                                       &  \\
 &    & versionName {\color[HTML]{009901}''1.0''}                                                   &  \\
 & \} &                                                                     & \\
\hline
\end{tabular}
\caption[Excerto da configuração padrão da aplicação móvel relativo às versões]{Excerto da configuração padrão da aplicação móvel relativo às versões}
\label{t:defaultconfigversions}
\end{table}

\begin{table}[H]
\centering
\begin{tabular}{|llll|}
\hline
 & \multicolumn{2}{l}{dependencies \{}                                     &  \\
 &    &  compile fileTree({\color[HTML]{009901} include}: [{\color[HTML]{009901} '*.jar'}], {\color[HTML]{009901} dir}: {\color[HTML]{009901} 'libs'})   &  \\
 &    & compile {\color[HTML]{009901} 'com.android.support:appcompat-v7:25.3.1'} &  \\
 &    & compile {\color[HTML]{009901} 'com.android.support:support-v4:25.3.1'} &  \\
 &    & compile {\color[HTML]{009901} 'com.android.support:design:25.3.1'} &  \\
 &    & compile files( {\color[HTML]{009901} 'libs/biolib.sdk.jar'})&  \\
 &    & compile files( {\color[HTML]{009901} 'libs/achartengine-1.1.0.jar'})&  \\
 & \} &                                                                     & \\
\hline
\end{tabular}
\caption[Excerto da configuração padrão da aplicação móvel relativo às dependências]{Excerto da configuração padrão da aplicação móvel relativo às dependências}
\label{t:defaultconfigdependency}
\end{table}

Relativamente às dependências temos que ter em atenção que existem duas dependências em especifico que tiveram que ser importadas manualmente. Uma delas é um \gls{SDK} para android (BioLib) \footnote{http://www.sdk.vitaljacket.com/} que disponibiliza uma biblioteca para a conexão a um dispositivo VitalJacket e facilita o processamento e a receção dos dados pela aplicação. Uma outra biblioteca que foi utilizada foi o ''AChartEngine'' \footnote{https://github.com/ddanny/achartengine} que é uma biblioteca para android que permite o desenho de gráficos.




\subsection{Tecnologias integradas e fontes de dados}
A aplicação móvel está a utilizar as \gls{API}s \gls{REST} definidas anteriormente em \ref{l:restapiAUTH} e em \ref{l:restapiRESOURCES}. Utiliza o serviço de autenticação e autorização para todas as interações envolvidas com registo, \textit{login} e acesso aos dados. O serviço de recursos é utilizado para guardar todas as medições efetuadas e também para as consultar posteriormente.\par 
Os dados estão a ser recolhidos através do \textit{VitalJacket} que é um dispositivo médico que conjuga a tecnologia têxtil com soluções avançadas de engenharia biomédica. O dispositivo permite a configuração para adquirir diferentes sinais vitais tais como o \gls{ECG}, frequência cardíaca e o acelerómetro. A comunicação entre o dispositivo móvel e o sensor é efetuada utilizando o \gls{SDK} para \textit{android} da biblioteca da BioLib utilizando o perfil de bluetooth \gls{SPP}.



\subsection{Interações suportadas }
Esta aplicação móvel foi desenvolvida com o objetivo principal de efetuar diferentes sessões de medições de dados identificando a relação temporal com a atividade física, todas as funcionalidades que tínhamos como objetivo implementar foram implementadas.


\par
\textbf{Menu Lateral de Navegação}
\par
Este menu lateral de navegação permite a um participante efetuar o registo ou efetuar o \textit{Login} na aplicação caso já tenha criado uma conta anteriormente.

\par
\textbf{Atividade Principal}
\par
O participante antes de efetuar \textit{Login} tem a atividade Principal praticamente vazia. Pode aceder ao menu lateral de navegação, ou às configurações (Figura \ref{f:seesettings}).
\begin{figure}[H]
\centering
\includegraphics[height=0.3\textwidth]{imgs/seesettings.png}
\caption[Opção na atividade principal para abrir as configurações]{Opção na atividade principal para abrir as configurações}
\label{f:seesettings}
\end{figure}
Quando o participante efetua o \textit{Login} na aplicação as suas sessões são carregadas e são listadas para que se possa verificar as medições efetuadas correspondentemente a cada uma. Na Figura \ref{f:afterlogin} podemos ver do lado esquerdo a situação em que o participante ainda não tem sessões de medições efetuadas e do lado direito quando já efetuou duas sessões de medições.

\begin{figure}[H]
\centering
\includegraphics[height=0.4\textwidth]{imgs/afterlogin.png}
\caption[Atividade Principal depois do \textit{Login} Efetuado]{Atividade Principal depois do \textit{Login} Efetuado}
\label{f:afterlogin}
\end{figure}

\par
\textbf{Atividade de Configurações}
\par
Na atividade de configurações (Figura \ref{f:settings}) temos a possibilidade de: escolher o dispositivo que vai ser utilizado e ao qual se irá conectar; escolher tanto os tipos de dados que são para recolher como a sua relação temporal em relação à atividade física na altura que a medição está a ser efetuada.
\begin{figure}[H]
\centering
\includegraphics[height=0.3\textwidth]{imgs/settings.png}
\caption[Menu de Configurações e as respetivas opções de configuração]{Menu de Configurações e as respetivas opções de configuração}
\label{f:settings}
\end{figure}

\par
\textbf{Atividade de Registo}
\par
O participante pode ainda não ter uma conta criada, mas a possibilidade de criar uma conta existe na aplicação e está disponível através do menu lateral. No processo de criação de uma nova conta o participante tem que escolher um utilizador que ainda não tenha sido utilizado e escolher uma \textit{password} para essa conta de utilizador(Figura \ref{f:createaccount}).
\begin{figure}[H]
\centering
\includegraphics[height=0.3\textwidth]{imgs/signup_app.png}
\caption[Atividade para registo de um novo participante]{Atividade para registo de um novo participante}
\label{f:createaccount}
\end{figure}

\par
\textbf{Atividade para efetuar o \textit{Login}}
\par
Esta atividade (Figura \ref{f:loginaccount}) pode ser chamada de duas duas maneiras diferentes: ao iniciar uma nova sessão de medições(é verificado que não tem \textit{login} efetuado); escolhendo a opção para efetuar \textit{Login} no menu lateral. É neste momento que o servidor de autorização e autenticação devolve um \textit{token} de acesso.
\begin{figure}[H]
\centering
\includegraphics[height=0.3\textwidth]{imgs/before_login.png}
\caption[Atividade para efetuar o \textit{Login}]{Atividade para efetuar o \textit{Login}}
\label{f:loginaccount}
\end{figure}

\par
\textbf{Atividade associada a uma nova sessão de leitura}
\par
Depois de efetuar o \textit{login} o participante pode então criar novas sessões de medições (Figura \ref{f:onNewSession}), pode parar quando quiser e irá voltar a visualizar a sua lista de medições com o acréscimo da última sessão efetuada. 
\begin{figure}[H]
\centering
\includegraphics[height=0.3\textwidth]{imgs/reading.png}
\caption[Atividade relativa a uma nova sessão de medições]{Atividade relativa a uma nova sessão de medições}
\label{f:onNewSession}
\end{figure}

\par
\textbf{Atividade Relativa a uma sessão}
\par
Depois do participante ter entrado na aplicação com a sua conta pode visualizar as medições relativas a uma sessão. Para isso basta clicar numa das sessões que compõem a lista. Nesta atividade (Figura \ref{f:read_data}) pode visualizar os dados relativos a cada tipo de medição, entre eles a frequência cardíaca, \gls{ECG} e acelerómetro. Para isso tem disponíveis três diferentes separadores para mudar de tipo de medição. 
\begin{figure}[H]
\centering
\includegraphics[height=0.4\textwidth]{imgs/read_data.png}
\caption[Atividade para visualizar os dados relacionados a uma determinada sessão]{Atividade para visualizar os dados relacionados a uma determinada sessão}
\label{f:read_data}
\end{figure}

\section{Implementação da aplicação de revisão}
\subsection{Tecnologias integradas e fontes de dados}

A aplicação \textit{web} de revisão está a utilizar as \gls{API}s \gls{REST} definidas anteriormente em \ref{l:restapiAUTH} e em \ref{l:restapiRESOURCES}. Utiliza o serviço de autenticação e autorização para todas as interações envolvidas com registo, \textit{login} e acesso aos dados. O serviço de recursos é utilizado para consultar todas as medições efetuadas pelos participantes em estudo e para guardar os dados demográficos de cada participante adicionado ao seu estudo.\par

\subsection{Componentes da aplicação}
Esta aplicação foi desenvolvida como aplicação \textit{web} com recurso a \gls{HTML}, \textit{Java Script} e \textit{\gls{CSS}}. Foi utilizada a plataforma \textit{Bootstrap} \cite{bootstrap} da versão 3.3.7 que fornece uma base em \gls{HTML} e \gls{CSS} muito popular no mundo das aplicações \textit{web} para desenvolvimento responsivo adaptável a plataformas móveis. \par
Para a comunicação com a \gls{API} \gls{REST} foi utilizado o \textit{jQuery} \cite{jquery}.
Relativamente à visualização dos dados para cada sessão foi utilizado a biblioteca \textit{HighCharts} \cite{highcharts} para a visualização dos dados relacionados com o acelerómetro e \gls{ECG}. Para a frequência cardíaca decidimos utilizar um visualizador padrão desenvolvido pela \gls{OMH} denominado de \textit{Open mHealth Web Visualizations} \cite{omhwebvisualizations}.

\subsection{Interações suportadas }


Esta aplicação \textit{web} foi desenvolvida com o objetivo principal de efetuar a revisão dos dados inseridos pelos participantes. É então possível adicionar vários participantes e verificar as várias sessões de medições de dados.
\par
\textbf{Página Principal}
\par
A página principal tem uma secção de boas vindas e ainda permite o registo e a entrada na aplicação (Figura \ref{f:web-home}).

\begin{figure}[H]
\centering
\includegraphics[width=0.9\textwidth]{imgs/home.png}
\caption[Página principal da aplicação de revisão]{Página principal da aplicação de revisão}
\label{f:web-home}
\end{figure}

\par
\textbf{Página de Registo}
\par
O revisor pode ainda não ter uma conta criada, por isso antes de entrar na aplicação terá que efetuar o registo de uma nova conta. Na criação é necessário escolher uma nova conta que ainda não tenha sido utilizada acompanhada de uma password (Figura \ref{f:web-registo}).

\begin{figure}[H]
\centering
\includegraphics[width=0.95\textwidth]{imgs/signup_web.png}
\caption[Página de registo na aplicação de revisão]{Página de registo na aplicação de revisão}
\label{f:web-registo}
\end{figure}

\par
\textbf{Página para efetuar o \textit{Login}}
\par
Esta página (Figura \ref{f:web-login}) pode ser chamada de duas maneiras diferentes. Ao clicar nos participantes da secção de boas vindas é detetado que o \textit{login} ainda não foi efetuado e é feito o redirecionamento para esta página, ou então ao clicar no botão Entrar na página principal. Depois de efetuar o \textit{login} pode visualizar os seus participantes.

\begin{figure}[H]
\centering
\includegraphics[width=0.8\textwidth]{imgs/login_web.png}
\caption[Página de \textit{login} na aplicação de revisão]{Página de \textit{login} na aplicação de revisão}
\label{f:web-login}
\end{figure}

\par
\textbf{Página com todos os participantes}
\par
Esta página é composta por uma grelha onde pode visualizar todos os seus participantes e também adicionar novos participantes. Assim que são adicionados novos participantes a grelha vai ficando mais composta, como na Figura \ref{f:gridParticipante}

\begin{figure}[H]
\centering
\includegraphics[width=0.8\textwidth]{imgs/list_after_add_web.png}
\caption[Página com grelha de participantes]{Página com grelha de participantes}
\label{f:gridParticipante}
\end{figure}


\par
\newpage
\par
\textbf{Página para adicionar novo participante}
\par
Nesta página (Figura \ref{f:web-newparticipante}) pode adicionar um novo participante, para isso tem que identificar o \textit{username} do participante e todos os dados demográficos disponíveis deste participante. Depois de adicionar vai ser redirecionado para a página relativa a todos os participantes. 

\begin{figure}[H]
\centering
\includegraphics[width=1\textwidth]{imgs/add_participant_web.png}
\caption[Página para adicionar novo participante ao estudo]{Página para adicionar novo participante ao estudo}
\label{f:web-newparticipante}
\end{figure}
\newpage

\par
\textbf{Página com leituras do participante}
\par
Nesta página pode visualizar todos os dados demográficos e fisiológicos do participante. Os dados demográficos podem ser atualizados e os dados fisiológicos que podem ser consultados são todos aqueles inseridos pelo participante. Para isso terá que escolher um intervalo de datas para procurar por sessões de leitura nesse intervalo, depois disso as várias sessões compõem um \textit{dropdow menu} (Figura \ref{f:webleiturassessoes}) onde pode escolher a sessão que pretende rever. 

\begin{figure}[H]
\centering
\includegraphics[width=1\textwidth]{imgs/user-info-web.png}
\caption[Página com dados demográficos e fisiológicos do participante]{Página com dados demográficos e fisiológicos do participante}
\label{f:webleiturassessoes}
\end{figure}

Podemos então visualizar os dados em gráficos relativos a cada tipo de dados recolhido (Figura \ref{f:leiturasdados}). 

\begin{figure}[H]
\centering
\includegraphics[width=1\textwidth]{imgs/dadosfisiologicos-web.png}
\caption[Dados fisiológicos do participante numa determinada sessão]{Dados fisiológicos do participante numa determinada sessão}
\label{f:leiturasdados}
\end{figure}
 Relativamente a cada tipo de dados pode ser efetuado o \textit{zoom} para em cada um dos gráficos, e ainda no caso do acelerómetro escolher os dados a apresentar, neste caso foi seleccionada apenas a aceleração (Figura \ref{f:withzoom}).
 
 \begin{figure}[H]
\centering
\includegraphics[width=1\textwidth]{imgs/withzoom-choosen-web.png}
\caption[Dados fisiológicos do participante mais detalhados]{Dados fisiológicos do participante mais detalhados}
\label{f:withzoom}
 \end{figure}
\section{Como aplicar e estender a plataforma em novos contextos}

% explicar como se pega no que ficou feito para usar num novo projeto e como se pode estender (e.g: para novos tipos de dados)

\subsection{Criar um novo tipo de dados}

A criação de um novo tipo de dados pode ser feita com a ajuda do repositório \textit{schemas} \cite{schemas-rep} que foi criado pela organização da \gls{OMH}. Para isto, o desenvolvimento do novo esquema de dados deve ser feito em \gls{JSON} \textit{schema}. Vou descrever agora a lista de tarefas para conseguir efetuar a criação de um novo tipo de dados. Para este tutorial vou criar o tipo de dados acelerómetro. É um dos tipos de dados que são obtidos através do \textit{VitalJacket}.

\begin{enumerate}
  \item Clonar o repositório correspondente para um diretório à sua escolha 
  De seguida vamos fazer duas coisas principais: uma delas é criar o ficheiro que define o novo tipo de dados; a outra é criar um ficheiro com uma amostra do novo tipo de dados.
  \item Criar um ficheiro que define o tipo de dados em \gls{JSON} \textit{schema}, para isso, adicione um novo ficheiro no diretório \textit{schema/omh}. O nome deste ficheiro tem que ser composto pelo nome que quer dar ao tipo de dados e a versão correspondente, tem que terminar com a extensão \textit{.json}. No meu caso criei com o nome \textit{accelerometer-1.0.json}  \par A versão escolhida aqui é a 1.0 mas podia ser qualquer outra. Pode reparar que no diretório \textit{schema/omh} tem acesso a todos os ficheiros que definem todos os tipos de dados.
  \item Adicionar uma nova pasta ao diretório \textit{testdata/omh} como o nome respetivo ao ficheiro criado anteriormente. Neste caso criar uma pasta com o nome \textit{accelerometer}.
  \item Vai agora criar uma pasta correspondente à versão introduzida no ponto 2. Neste caso é 1.0 e uma outra pasta dentro da criada anteriormente com o nome \textit{shouldPass}.
  \item Dentro do diretório \textit{shouldPass} vai ter que criar um ou vários ficheiros para ser utilizados como amostras do tipo de dados. O objetivo deste diretório é ter várias amostras válidas testando-as com o tipo de dados criado no ponto 2. Para este tutorial criei o ficheiro \textit{example.json}. \par Neste ponto o nome do ficheiro é opcional só tem que acabar com a extensão \textit{.json}\par
  Como criou o diretório \textit{shouldPass}, pode também criar o diretório \textit{shouldFail} criando também vários ficheiros para testar o tipo de dados criado. \par 
  Neste ponto tem tudo preparado para começar a criar o novo tipo de dados e validar a amostra com a nova definição do novo tipo de dados criados.
  
  \begin{figure}[!ht]
  \centering
  \includegraphics[width=0.5\textwidth]{imgs/newsampledata.png}
  \caption[Esquema de diretório para o exemplo do novo tipo de dados]{Esquema de diretório para o exemplo do novo tipo de dados}
  
  \label{f:directorynewsample}
\end{figure}
  
\item Vamos agora preencher o ficheiro com a definição do novo tipo de dados, aquele que criou no ponto 2. O ficheiro vai ser criado no formato de \gls{JSON} \textit{schema}. Para suportar a criação deste ficheiro pode reutilizar \textit{schemas} existentes \cite{schema-library} e referênciá-los, deste modo estará a criar um modelo com tipos de dados normalizados. No meu caso vou reutilizar um \textit{schema} para definir a data/hora e um outro para definir a relação temporal relativa à atividade física no momento da leitura. Os restantes dados estão relacionados com o acelerómetro em si, a sessão e a leitura respetiva.
O ficheiro fica do seguinte modo: 

\begin{figure}[H]
\inputminted[fontsize=\scriptsize]{json}{code/accelerometer-1.0.json}
\caption[\gls{JSON} \textit{schema} para o novo tipo de dados de acelerómetro]{\gls{JSON} \textit{schema} para o novo tipo de dados de acelerómetro}
\label{f:accelerometer-json-schema}
\end{figure}

\item Preencha o ficheiro com uma amostra do novo tipo de dados, para isso tem que ter em conta a definição utilizada, porque se esta amostra não for compatível não vai passar no validador.

\begin{figure}[H]
\inputminted[fontsize=\scriptsize]{json}{code/example.json}
\caption[Exemplo do tipo de dados de acelerómetro]{Exemplo do tipo de dados de acelerómetro}
\label{f:accelerometer-json-data}
\end{figure}

\item Para compilar e executar o validador tem que executar o comando \textit{./gradlew test-data-validator:bootRun} no diretório principal ''schemas'' ou então execute o comando \textit{./gradlew bootRun} no diretório \textit{schemas/test-data-validator}. 

\subsection{Como utilizar o novo tipo de dados criado}

Para adicionar o novo tipo de dados criado, tem que o integrar com os restantes tipos de dados existentes. \par Tendo em conta que já tem o repositório omh-dsu-ri clonado através do comando ''git clone https://github.com/luistduarte/omh-dsu-ri', terá que copiar o seu novo tipo de dados para o diretório ''omh-dsu-ri/resource-server/src/main/resources/schema/omh'' que é onde se encontram os restantes tipos de dados válidos.
A partir deste momento a inserção de novos dados compatíveis com o \textit{data schema} criado já é possível, para isso na criação e inserção de \textit{datapoints} é necessário especificar no cabeçalho o tipo de dados que está a ser inserido, para este ser validado e corretamente criado.
\end{enumerate}

\cleardoublepage
