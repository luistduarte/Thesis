\chapter{Resultados}
\section{Protótipo integrado}
%recapitular o que se pode fazer com o protótipo que foi construído e como é que ele está apto para suportar experiências de I&D

O protótipo que foi construído tem a capacidade de suportar experiências de \gls{ID} pois tem a capacidade de gerir contas de utilizadores participantes e revisores que estejam envolvidos num projeto de \gls{ID}. Fornece a possibilidade de criação (inserção) de novas medições por parte dos participantes e ainda consultar estas medições inseridas. Os revisores podem consultar todas as medições inseridas pelos participantes envolvidos no seu estudo.\par
O servidor de recursos suporta extensibilidade dos dados sem que se tenha de alterar o esquema da base de dados o que é uma grande vantagem. Este servidor utiliza ainda o servidor de autorização e autenticação para permitir o acesso aos pedidos recebidos. \par
Este protótipo tem então todas as características de uma plataforma \textit{mHealth}, ou seja, permite que exista um participante a recolher dados de dispositivos médicos, encaminhados através do \textit{smartphone}, e que estes estejam a ser monitorizados pelo revisor do estudo sem que eles estejam próximos um do outro, dado que todos os dados recolhidos são guardados no servidor.

\section{Adequação das plataformas estudadas}

% discutir de forma crítica o estado de prontidão ou falta dela, das plataformas existentes para criar aplicações de mHealth como a que foi desenvolvida
Na fase exploratória conseguimos encontrar duas plataformas que serviam para ser utilizadas como backend do nosso protótipo, para além do \gls{OMH} que foi a plataforma utilizada conseguimos também concluir que o \gls{FHIR} servia para as necessidades existentes. O \textit{Google Fit} foi logo excluído pois após um estudo um pouco mais aprofundado percebemos que relativamente às propriedades do tipo de dados, não suporta vetores, ficando logo praticamente excluído como possível escolha para o \textit{backend}.\par 
Apesar de termos duas plataformas que poderiam ser utilizadas como \textit{backend} nenhuma delas se encontrava pronta para utilização sem qualquer adaptação. O \gls{FHIR} por um lado não suportava gestão de identidades e não controlava o acesso aos pedidos recebidos. Relativamente ao \gls{OMH} o que acontecia é que os dados não eram validados, e os dados inseridos eram proprietários, ou seja, apenas a pessoa que fazia a inserção dos dados recolhidos os podia voltar a consultar, isto porque era apenas utilizado o \textit{token} de acesso para identificar de que pessoa seria os dados consultados.
Podemos concluir aqui que, nenhuma das plataformas estudadas podia ser utilizada para o desenvolvimento de aplicações \textit{mHealth} sem qualquer tipo de adaptação, o que não impossibilita a sua utilização após algumas adaptações pois até são convenientemente completas.

\section{Recomendações para o desenvolvimento de aplicações de mHealth}

% ace ao trabalho desenvolvido, que recomendações práticas podem ser apresentadas para o desenvolvimento de aplicações de mHealth, especialmente tendo em conta o backend e a utilizaçãod e soluções padronizadas?

Face ao trabalho desenvolvido, as recomendações que podemos dar é que a plataforma do \gls{OMH} é bastante apelativa e de possível solução para um backend de uma aplicação \textit{mHealth}. Tem uma boa quantidade de esquemas de dados já disponíveis e é de fácil extensibilidade. Caso a aplicação seja de pequena dimensão esta solução é prática, pois facilmente colocamos o \textit{backend} num servidor, podendo inserir dados e consultar os mesmos. Quando estamos a falar do desenvolvimento de uma aplicação para uma clínica e ou hospital, já não recomendamos o \gls{OMH}, pois o \gls{FHIR} é bastante mais completo e serviria muito melhor as necessidades, sendo também mais fácil a posterior exportação e importação de dados clínicos e ou hospitalares. \par
Ao utilizar o resultado desenvolvido neste trabalho terá o benefício de começar com um \textit{backend} mais robusto e completo, terá a certeza que os dados inseridos respeitam o \gls{JSON} \textit{schema} associado o que é uma vantagem para uma posterior consulta. Terá ainda a possibilidade de integrar o sistema com módulos externos permitindo uma análise sobre os dados recolhidos.


\cleardoublepage