\chapter{Resultados}
\section{Protótipo integrado}
%recapitular o que se pode fazer com o protótipo que foi construído e como é que ele está apto para suportar experiências de I&D

O protótipo que foi construído tem a capacidade de suportar experiências de \gls{ID} pois, tem a capacidade de gerir contas de utilizadores participantes e revisores que estejam envolvidos num projeto de \gls{ID}, fornece a possibilidade de criação (inserção) de novas medições e a consulta das mesmas, pelo participante as dele próprio e pelo revisor as de todos os participantes envolvidos no seu estudo.\par
O servidor de recursos suporta extensibilidade dos dados sem que se tenha de alterar o esquema da base de dados o que é uma grande vantagem. Este servidor utiliza ainda o servidor de autorização e autenticação para permitir o acesso aos pedidos recebidos. \par
Este protótipo tem então todas as características de uma plataforma \textit{mHealth}, ou seja, permite que exista um participante a recolher dados e que estes estejam a ser monitorizados pelo revisor do estudo sem que eles estejam próximos um do outro, dado que todos os dados recolhidos são guardados no servidor.

\section{Adequação das plataformas estudadas}

% discutir de forma crítica o estado de prontidão ou falta dela, das plataformas existentes para criar aplicações de mHealth como a que foi desenvolvida
Na fase exploratória conseguimos encontrar duas plataformas que serviam para ser utilizado como backend do nosso protótipo, para além do \gls{OMH} que foi a plataforma utilizada conseguimos também concluir que o \gls{FHIR} servia para as necessidades existentes. O problema é que nenhuma delas se encontrava pronta para utilização sem qualquer adaptação. O \gls{FHIR} por um lado não suportava gestão de identidades e não controlava o acesso aos pedidos recebidos. Relativamente ao \gls{OMH} o que acontecia é que os dados não eram validados, e os dados inseridos eram proprietários, ou seja, apenas a pessoa que fazia a inserção dos dados recolhidos os podia voltar a consultar, isto porque era apenas utilizado o \textit{token} de acesso para identificar de que pessoa seria os dados consultados.

\section{Recomendações para o desenvolvimento de aplicações de mHealth}

% ace ao trabalho desenvolvido, que recomendações práticas podem ser apresentadas para o desenvolvimento de aplicações de mHealth, especialmente tendo em conta o backend e a utilizaçãod e soluções padronizadas?

Face ao trabalho desenvolvido, as recomendações que podemos dar é que a plataforma do \gls{OMH} é bastante apelativa e de possível solução para um backend de uma aplicação \textit{mHealth}. Tem uma boa quantidade de esquemas de dados já disponíveis e é de fácil extensibilidade. Caso a aplicação seja de pequena dimensão esta solução é prática, pois facilmente colocamos o \textit{backend} num servidor, podendo inserir dados e consultar os mesmos. Quando estamos a falar do desenvolvimento de uma aplicação para uma clínica e ou hospital, já não recomendamos o \gls{OMH}, pois o \gls{FHIR} é bastante mais completo e serviria muito melhor as necessidades, sendo também mais fácil a posterior exportação e importação de dados clínicos e ou hospitalares. 

\cleardoublepage