\chapter{Introdu\c c\~ao}

\section{Motivação}

A grande disponibilidade de dispositivos móveis e a pressão nos sistemas de saúde levaram a uma explosão do número de aplicações móveis para a saúde (\textit{mHealth}). Estas aplicações são cada vez mais relevantes para monitorização e acompanhamento de doentes, especialmente com doenças crónicas, mas, quando se trata de escolher arquiteturas de sistema para o seu desenvolvimento, não há respostas óbvias. Surge então a necessidade de procurar e escolher um backend para aplicações \textit{mHealth} com o objetivo de se guardar e posteriormente consultar dados fisiológicos e demográficos dos doentes.
Dentro dos backends existentes é necessário encontrar um que esteja apto para ser utilizado e que consiga dar suporte a uma aplicação \textit{mHealth}, ou perceber as alterações que seriam necessárias para que fosse possível utilizar um destes backends disponíveis. Uma das características a ter em conta ao escolher um destes backends disponíveis é perceber se os dados se encontram normalizados para uma possível exportação ou importação de dados clínicos ou hospitalares para uma possível integração num sistema externo. \par
No \gls{IEETA}, a unidade que acolheu este trabalho, a atividade de \gls{ID} é constante e o desenvolvimento de aplicações móveis é frequente, e muitas vezes integrado em cenários com experiências que recolhem parâmetros fisiológicos. A identificação de recomendações práticas para a engenharia do backend de uma aplicação de \textit{mHealth} tem também interesses práticos para as atividades deste instituto.

\section{Objetivos}
Para o desenvolvimento desta dissertação temos como objetivo principal encontrar um backend que esteja apto a ser utilizado, ou que o possamos utilizar após algumas adaptações, para servir como suporte a uma aplicação \textit{mHealth}, explorando e estudando as plataformas existentes, para perceber o que é que cada uma delas tem para nos oferecer. \par
Como segundo objetivo, propomo-nos desenvolver um sistema integrado, servindo como prova de conceito, utilizando um problema de Investigação e Desenvolvimento que irá gerir a informação fisiológica de pessoas através de uma aplicação móvel. Estas pessoas são participantes de um estudo, podendo estas ser monitorizadas remotamente por especialistas através de uma aplicação web. Como suporte destas aplicações deverá ser utilizado o backend escolhido no primeiro objetivo.
\cleardoublepage