\chapter{Introdu\c c\~ao}

\section{Motivação}

A grande disponibilidade de dispositivos móveis e a pressão nos sistemas de saúde levaram a uma explosão do número de aplicações móveis para a saúde (\textit{mHealth}). Estas aplicações são cada vez mais relevantes para monitorização e acompanhamento de doentes, especialmente com doenças crónicas\cite{mHealth-chronic-disease}, mas, quando se trata de escolher arquiteturas de sistema para o seu desenvolvimento, não há respostas óbvias. Surge então a necessidade de procurar e escolher um \textit{backend} para aplicações \textit{mHealth} com o objetivo de se guardar e posteriormente consultar dados fisiológicos e demográficos dos doentes.
Dentro dos \textit{backends} existentes é necessário encontrar um que esteja apto para ser utilizado e que consiga dar suporte a uma aplicação \textit{mHealth}, ou perceber as alterações que seriam necessárias para que fosse possível utilizar um destes \textit{backends} disponíveis. Uma das características a ter em conta ao escolher um destes \textit{backends} disponíveis é perceber se os dados se encontram normalizados para uma possível exportação ou importação de dados clínicos ou hospitalares para uma possível integração num sistema externo. \par
No \gls{IEETA}, a unidade que acolheu este trabalho, a atividade de \gls{ID} é constante e o desenvolvimento de aplicações móveis é frequente, e muitas vezes integrado em cenários com experiências que recolhem parâmetros fisiológicos. A identificação de recomendações práticas para a engenharia do \textit{backend} de uma aplicação de \textit{mHealth} tem também interesse prático para as atividades deste instituto.

\section{Objetivos}
Para o desenvolvimento desta dissertação temos como objetivo principal encontrar um \textit{backend} que esteja apto a ser utilizado, ou que o possamos utilizar após algumas adaptações, para servir como suporte a aplicações em \textit{mHealth}, explorando e estudando as plataformas existentes, para perceber o que cada uma delas tem para nos oferecer. \par
Como segundo objetivo, propomo-nos desenvolver um sistema integrado, servindo como prova de conceito, utilizando um problema típico de um projeto de \gls{ID}, que irá gerir a informação fisiológica de pessoas através de uma aplicação móvel. Estas pessoas são participantes de um estudo, podendo estas ser monitorizadas remotamente por especialistas através de uma aplicação web. Como suporte destas aplicações deverá ser utilizado o \textit{backend} escolhido no primeiro objetivo.

\section{Estrutura da dissertação}
Esta dissertação está organizada por 8 capítulos.\par 
No capítulo 1, apresentamos a motivação e os objetivos para o desenvolvimento da dissertação.\par 
No capítulo 2, revemos alguns conceitos e o estado da arte relativamente às aplicações \textit{mHealth} e arquiteturas relacionadas. Apresentamos ainda algumas plataformas de \textit{backend} que podem vir a ser utilizados neste tipo de aplicações, realizando uma análise comparativa entre eles.\par 
No capítulo 3, descrevemos os casos de uso do sistema a desenvolver, partindo de um cenário concreto extraindo os requisitos.\par 
No capítulo 4, é feita uma avaliação exploratória das várias soluções disponíveis para o \textit{backend} de uma aplicação \textit{mHealth}. Neste capítulo foi contemplado as dificuldades encontradas e as adaptações necessárias relativamente a cada \textit{backend} elaborando uma comparação e concluindo qual o \textit{backend} a ser utilizado. \par
No capítulo 5, apresentamos uma possível arquitetura para a solução do sistema a desenvolver. \par 
No capítulo 6, descrevemos a implementação do sistema tendo em conta o \textit{backend} escolhido no capítulo 4, e a arquitetura proposta no capítulo 5. \par 
No capítulo 7, discutimos os resultados obtidos e damos algumas recomendações para o desenvolvimento de aplicações de \textit{mHealth}. \par 
No capítulo 8, apresentamos as conclusões obtidas e o trabalho futuro a ser desenvolvido.
\cleardoublepage