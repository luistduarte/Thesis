\chapter{Conclusão}

Neste trabalho, foi desenvolvida uma investigação de possíveis \textit{backends} para dar suporte a aplicações \textit{mHealth}. 
Começamos por selecionar três plataformas que se revelaram candidatas a ser utilizadas para este fim. Para se chegar a uma melhor conclusão baseada em factos práticos foi desenvolvido um conjunto de atividades exploratórias relativamente a cada uma das plataformas de \textit{backend} com o objetivo de aprender melhor o funcionamento, os pontos fortes e fracos de cada um. \par 
Chegamos então à conclusão que o \textit{Open mHealth} seria o melhor \textit{backend} dentro dos explorados para dar suporte às aplicações de \textit{mHealth}, para o cenário proposto.
Posto isto, foi então desenvolvido um sistema integrado utilizando um problema de \gls{ID} para servir como prova de conceito colocando à prova o \textit{backend} escolhido para concluir se este era ou não viável.
O sistema foi desenvolvido com sucesso e chegamos à conclusão que o \textit{Open mHealth} pode ser utilizado como \textit{backend} de aplicações \textit{mHealth} tendo em conta que foram apenas necessárias algumas adaptações e implentação extra com o objetivo de enriquecer a plataforma e a completar. \par
Esta investigação e exploração desenvolvida vem então concluir que o \textit{backend} do \textit{Open mHealth} é fiável e capaz de servir aplicações de \textit{mHealth}.


\section{Trabalho futuro}
Tendo em conta o trabalho desenvolvido durante esta dissertação, existem pontos a ser melhorados e complementados, entre eles podemos ter a criação de novos esquemas de dados. Ao estender o conjunto de esquemas de dados do \textit{backend} do \textit{Open mHealth} a plataforma ficará mais completa abrangendo cada vez mais todas as necessidades existentes para se guardar diferentes tipos de dados.\par
Foi criada a possibilidade de ligação de módulos externos através de um \textit{message bus}. Estes módulos podem subscrever um determinado tipo de dados para posterior análise. Não foi criado nenhum módulo completo para analisar os tipos de dados recolhidos, no entanto pode ser desenvolvido diferentes módulos com esse objetivo. Um módulo depois de efetuar a subscrição, todos os dados do tipo que foi subscrito vão ser reencaminhados para ele. Existe a possibilidade de diferentes módulos subscreverem o mesmo tipo de dados para efetuar diferentes análises.\par
Relativamente à segurança e proteção dos dados demográficos e fisiológicos dos participantes no caso de estudo pode também ser melhorado, garantindo a utilização de dados pseudo-anonimizados.
Tendo em conta que a comunicação com o \textit{backend} é toda feita através de pedidos \gls{HTTP}, a utilização deste \textit{backend} com uma aplicação iOS não será problema, mas podem ser desenvolvidos testes com diferentes dispositivos utilizando outros tipos de dados.

\cleardoublepage