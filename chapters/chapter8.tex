\chapter{Conclusão}
%Para o desenvolvimento desta dissertação temos como objetivo principal encontrar um backend que esteja apto a ser utilizado, ou que o possamos utilizar após algumas adaptações, para servir como suporte a uma aplicação \textit{mHealth}, explorando e estudando as plataformas existentes, para perceber o que é que cada uma delas tem para nos oferecer. \par

%Como segundo objetivo, propomo-nos desenvolver um sistema integrado, servindo como prova de conceito, utilizando um problema de Investigação e Desenvolvimento que irá gerir a informação fisiológica de pessoas através de uma aplicação móvel. Estas pessoas são participantes de um estudo, podendo estas ser monitorizadas remotamente por especialistas através de uma aplicação web. Como suporte destas aplicações deverá ser utilizado o backend escolhido no primeiro objetivo.

Tendo em conta os objetivos iniciais, foi desenvolvida uma investigação de possíveis \textit{backends} para dar suporte a aplicações \textit{mHealth}, depois disto foram selecionados três dos quais à primeira vista eram os fortes candidatos a ser utilizados para este fim. Depois disto foi então desenvolvido um conjunto de atividades exploratórias relativamente a cada um deles com o objetivo de aprender melhor o funcionamento, os pontos fortes e fracos de cada um. Chegamos então à conclusão que o \textit{Open mHealth} seria o melhor \textit{backend} dentro dos explorados para dar suporte às aplicações de \textit{mHealth}.
Posto isto, foi então desenvolvido um sistema integrado utilizando um problema de \gls{ID} para servir como prova de conceito colocando à prova o \textit{backend} escolhido para concluir se este era ou não viável.
O sistema foi desenvolvido com sucesso e chegamos à conclusão que o \textit{Open mHealth} pode ser utilizado como \textit{backend} de aplicações \textit{mHealth} tendo em conta que foram apenas necessárias algumas adaptações com o objetivo de enriquecer a plataforma e a completar.


\section{Trabalho futuro}
Tendo em conta o trabalho desenvolvido durante esta dissertação existe pontos a ser melhorados e complementados, entre eles podemos ter a criação de novos esquemas de dados. Ao estender o conjunto de esquemas de dados do backend do \textit{Open mHealth} a plataforma ficará mais completa abrangendo cada vez mais todas as necessidades existentes para se guardar diferentes tipos de dados.\par
Foi criada a possibilidade de ligação de módulos externos a um \textit{message bus}. Estes módulos podiam fazer a subscrição de um tipo de dados para posterior análise. Não foi criado nenhum módulo completo para analisar os tipos de dados recolhidos, no entanto pode ser desenvolvido diferentes módulos com esse objetivo.\par
Relativamente à segurança e proteção dos dados demográficos e fisiológicos dos participantes no caso de estudo pode também ser melhorado.

\cleardoublepage