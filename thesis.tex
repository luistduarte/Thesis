%
% uaThesis example (for a thesis written in Portuguese)
%
% the complete list of options and commands can be found in uaThesis.sty
%

\documentclass[11pt,twoside,a4paper]{report}
\usepackage[DETI,newLogo]{uaThesis}

\def\ThesisYear{2017}

% optional packages
\usepackage[portuguese]{babel}
\usepackage{hyperref}
\usepackage{amsmath}
\usepackage{amssymb}
\usepackage{xspace}% used by \sigla
\usepackage{natbib}
\usepackage[utf8]{inputenc}
\bibliographystyle{unsrtnat}
\setcitestyle{numbers}
\usepackage{setspace}
\usepackage{float}
\usepackage{enumitem}
\usepackage{minted}
\usepackage{diagbox}
\usepackage[table,xcdraw]{xcolor}
\usepackage{import}
\usepackage{tabularx}
\onehalfspacing

\usepackage{glossaries}
\makeglossaries

\usepackage{glossary-mcols}


% optional (comment to use default)s
%   depth of the table of contents
%     1 ... chapther and sections
%     2 ... chapters, sections, and subsections
%     3 ... chapters, sections, subsections, and subsubsections
\setcounter{tocdepth}{3}

% optional (comment to used default)
%   horizontal line to separate floats (figures and tables) from text
\def\topfigrule{\kern 7.8pt \hrule width\textwidth\kern -8.2pt\relax}
\def\dblfigrule{\kern 7.8pt \hrule width\textwidth\kern -8.2pt\relax}
\def\botfigrule{\kern -7.8pt \hrule width\textwidth\kern 8.2pt\relax}

% custom macros (could also be defined using \newcommand)
\def\I{\mathtt{i}}         % one possible way to represent $\sqrt{-1}$
\def\Exp#1{e^{2\pi\I #1}}  % argument inside braces, i.e., "{}"
\def\EXP#1.{e^{2\pi\I #1}} % argument finishes when a full stop is encountered, i.e., "."
\def\sigla{\LaTeX\xspace}  % use as "blabla \sigla blabla (no need to do "blabla \sigla\ blabla"

\def\AddVMargin#1{\setbox0=\hbox{#1}%
                  \dimen0=\ht0\advance\dimen0 by 2pt\ht0=\dimen0%
                  \dimen0=\dp0\advance\dimen0 by 2pt\dp0=\dimen0%
                  \box0}   % add extra vertical space above and below the argument (#1)
\def\Header#1#2{\setbox1=\hbox{#1}\setbox2=\hbox{#2}%
           \ifdim\wd1>\wd2\dimen0=\wd1\else\dimen0=\wd2\fi%
           \AddVMargin{\parbox{\dimen0}{\centering #1\\#2}}} % put #1 on top #2


\begin{document}

%
% Cover page (use only one of the first two \TitlePage)
%

% Second alternative, with a citation
\TitlePage
  %\GRID  % for debugging ONLY
  \HEADER{\BAR\FIG{}}
         {\ThesisYear}
  \TITLE{Lu\'is Tiago Marques \newline Duarte}
        {Como escrever uma tese bonita e cheia de resultados importantes}
\EndTitlePage
\titlepage\ \endtitlepage % empty page


%
% Initial thesis pages
%

\TitlePage
  \HEADER{}{\ThesisYear}
  \TITLE{Lu\'is Tiago Marques \newline Duarte}
        {Como escrever uma tese bonita e cheia de resultados importantes}
  \vspace*{15mm}
  \TEXT{}
       {Disserta\c c\~ao apresentada \`a Universidade de Aveiro para cumprimento dos requisitos necess\'arios \`a obten\c c\~ao do
        grau de Mestre em Engenharia de Computadores e Telem\'atica, realizada sob a orienta\c c\~ao cient\'ifica do Doutor Il\'idio Oliveira, Professor auxiliar do Departamento de Eletr\'onica, Telecomunica\c c\~oes e Inform\'atica da Universidade de Aveiro.}
\EndTitlePage
\titlepage\ \endtitlepage % empty page


\titlepage\ 

\vspace*{30mm}\begin{flushright}
Dedico este trabalho aos meus pais e amigos.
\end{flushright}
\endtitlepage

\titlepage\ \endtitlepage % empty page

\TitlePage
  \vspace*{55mm}
  \TEXT{\textbf{o j\'uri~/~the jury\newline}}
       {}
  \TEXT{presidente~/~president}
       {\textbf{ABC}\newline {\small
        Professor Catedr\'atico da Universidade de Aveiro (por delega\c c\~ao da Reitora da
        Universidade de Aveiro)}}
  \vspace*{5mm}
  \TEXT{vogais~/~examiners committee}
       {\textbf{DEF}\newline {\small
        Professor Catedr\'atico da Universidade de Aveiro (orientador)}}
  \vspace*{5mm}
  \TEXT{}
       {\textbf{GHI}\newline {\small
        Professor associado da Universidade J (co-orientador)}}
  \vspace*{5mm}
  \TEXT{}
       {\textbf{KLM}\newline {\small
        Professor Catedr\'atico da Universidade N}}
\EndTitlePage
\titlepage\ \endtitlepage % empty page

\TitlePage
  \vspace*{55mm}
  \TEXT{\textbf{agradecimentos~/\newline acknowledgements}}
       {\'E com muito gosto que aproveito esta oportunidade para agradecer a todos os que me
        ajudaram durante este longos e penosos anos, cheios de altos e baixos (mais baixos que
        altos)\ldots}
  \TEXT{}
       {Desejo tamb\'em pedir desculpa a todos que tiveram de suportar o meu desinteresse pelas
        tarefas mundanas do dia-a-dia, \ldots}
\EndTitlePage
\titlepage\ \endtitlepage % empty page

\TitlePage
  \vspace*{55mm}
  \TEXT{\textbf{Resumo}}
       {Nos dias que correm, \'e frequente um trabalho ser avaliado pela sua apar\^encia em vez de
        o ser pelo seu conte\'udo. Sendo assim, sem descurar este \'ultimo, nesta tese descrevemos
        maneiras revolucion\'arias de transformar um documento s\'olido e austero num documento
        s\'olido e belo, capaz de fazer chorar de alegria (ou de inveja) qualquer leitor, mesmo
        quando este n\~ao percebe nada do que l\'a est\'a escrito.}
  \TEXT{}
       {A explora\c c\~ao de novas descobertas na \'area da percep\c c\~ao visual, nomeadamente
        no que se refere \`a aprecia\c c\~ao de obras de arte geniais, \ldots}
\EndTitlePage
\titlepage\ \endtitlepage % empty page

\TitlePage
  \vspace*{55mm}
  \TEXT{\textbf{Abstract}}
       {Nowadays, it is usual to evaluate a work \ldots}
\EndTitlePage
\titlepage\ \endtitlepage % empty page


%
% Tables of contents, of figures, ...
%

\pagenumbering{roman}

\tableofcontents

\cleardoublepage
\listoffigures

\cleardoublepage
\listoftables

\cleardoublepage


\printglossary[style=mcolindex,title=Lista de Abrevia\c c\~oes e Acr\'onimos, nonumberlist]


\setacronymstyle{long-short}

\newacronym{oms}{OMS}{Organiza\c c\~ao Mundial de Sa\'ude}
\newacronym{mHealth}{mHealth}{Mobile Health}
\newacronym{ECG}{ECG}{Eletrocardiograma}
\newacronym{API}{API}{Application Programming Interface}
\newacronym{XML}{XML}{eXtensible Markup Language}
\newacronym{HTML}{HTML}{HyperText Markup Language}
\newacronym{HTTP}{HTTP}{Hypertext Transfer Protocol}
\newacronym{JSON}{JSON}{JavaScript Object Notation}
\newacronym{REST}{REST}{Representational State Transfer}
\newacronym{URI}{URI}{Uniform Resource Identifier}
\newacronym{CRUD}{CRUD}{Create, Read, Update and Delete}
\newacronym{OMH}{OMH}{Open mHealth}
\newacronym{HL7}{HL7}{Health Level Seven}
\newacronym{FHIR}{FHIR}{Fast Healthcare Interoperability Resources}
\newacronym{RFC}{RFC}{Request For Comments}
\newacronym{JVM}{JVM}{Java Virtual Machine}
\newacronym{RIM}{RIM}{Reference Information Model}
\newacronym{ID}{I\&D}{Investigação e Desenvolvimento}
\newacronym{DSU}{DSU}{Data Storage Unit}
\newacronym{SIG}{SIG}{Special Interest Group}
\newacronym{HDP}{HDP}{Health Device Profile}
\newacronym{SPP}{SPP}{Serial Port Profile}
% The chapters (usually written using the isolatin font encoding ...)

\cleardoublepage
\pagenumbering{arabic}


\import{chapters/}{chapter1.tex}

\import{chapters/}{chapter2.tex}

\import{chapters/}{chapter3.tex}

\import{chapters/}{chapter4.tex}

\import{chapters/}{chapter5.tex}

\import{chapters/}{chapter6.tex}

\import{chapters/}{chapter7.tex}

\import{chapters/}{chapter8.tex}


\renewcommand{\bibname}{Lista de Referências}
\bibliography{references}

\cleardoublepage

\end{document}
 