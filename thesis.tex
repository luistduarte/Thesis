%
% uaThesis example (for a thesis written in Portuguese)
%
% the complete list of options and commands can be found in uaThesis.sty
%

\documentclass[11pt,twoside,a4paper]{report}
\usepackage[DETI,newLogo]{uaThesis}

\def\ThesisYear{2017}

% optional packages
\usepackage[portuguese]{babel}
\usepackage{hyperref}
\usepackage{amsmath}
\usepackage{amssymb}
\usepackage{xspace}% used by \sigla
\usepackage{natbib}
\usepackage[utf8]{inputenc}
\bibliographystyle{unsrtnat}
\setcitestyle{numbers}
\usepackage{setspace}
\usepackage{float}
\usepackage{enumitem}
\usepackage{minted}
\usepackage{diagbox}
\usepackage[table,xcdraw]{xcolor}
\usepackage{import}
\usepackage{tabularx}
\onehalfspacing

\usepackage{glossaries}
\makeglossaries

\usepackage{glossary-mcols}


% optional (comment to use default)s
%   depth of the table of contents
%     1 ... chapther and sections
%     2 ... chapters, sections, and subsections
%     3 ... chapters, sections, subsections, and subsubsections
\setcounter{tocdepth}{2}

% optional (comment to used default)
%   horizontal line to separate floats (figures and tables) from text
\def\topfigrule{\kern 7.8pt \hrule width\textwidth\kern -8.2pt\relax}
\def\dblfigrule{\kern 7.8pt \hrule width\textwidth\kern -8.2pt\relax}
\def\botfigrule{\kern -7.8pt \hrule width\textwidth\kern 8.2pt\relax}

% custom macros (could also be defined using \newcommand)
\def\I{\mathtt{i}}         % one possible way to represent $\sqrt{-1}$
\def\Exp#1{e^{2\pi\I #1}}  % argument inside braces, i.e., "{}"
\def\EXP#1.{e^{2\pi\I #1}} % argument finishes when a full stop is encountered, i.e., "."
\def\sigla{\LaTeX\xspace}  % use as "blabla \sigla blabla (no need to do "blabla \sigla\ blabla"

\def\AddVMargin#1{\setbox0=\hbox{#1}%
                  \dimen0=\ht0\advance\dimen0 by 2pt\ht0=\dimen0%
                  \dimen0=\dp0\advance\dimen0 by 2pt\dp0=\dimen0%
                  \box0}   % add extra vertical space above and below the argument (#1)
\def\Header#1#2{\setbox1=\hbox{#1}\setbox2=\hbox{#2}%
           \ifdim\wd1>\wd2\dimen0=\wd1\else\dimen0=\wd2\fi%
           \AddVMargin{\parbox{\dimen0}{\centering #1\\#2}}} % put #1 on top #2


\begin{document}

%
% Cover page (use only one of the first two \TitlePage)
%

% Second alternative, with a citation
\TitlePage
  %\GRID  % for debugging ONLY
  \HEADER{\BAR\FIG{}}
         {\ThesisYear}
      \TITLE{\textbf{Lu\'is Tiago Marques \newline Duarte}}{\textbf{\textsf{Comparação e implementação de plataformas de \textit{backend} para soluções \textit{mHealth}}}}
      \TITLE{}{}
      \TITLE{}{}
  \TITLE{}
        {\textbf{\textsf{Assessment and implementation of backend frameworks for mHealth applications}}}

\EndTitlePage
\titlepage\ \endtitlepage % empty page


%
% Initial thesis pages
%

\TitlePage
  \HEADER{}{\ThesisYear}
        \TITLE{\textbf{Lu\'is Tiago Marques \newline Duarte}}{\textsf{\textbf{Comparação e implementação de plataformas de \textit{backend} para soluções \textit{mHealth}}}}
      \TITLE{}{}
      \TITLE{}{}
  \TITLE{}
        {\textsf{\textbf{Assessment and implementation of backend frameworks for mHealth applications}}}
  \vspace*{15mm}
  \TEXT{}
       {Disserta\c c\~ao apresentada \`a Universidade de Aveiro para cumprimento dos requisitos necess\'arios \`a obten\c c\~ao do
        grau de Mestre em Engenharia de Computadores e Telem\'atica, realizada sob a orienta\c c\~ao cient\'ifica do Doutor Il\'idio Castro Oliveira, professor auxiliar do Departamento de Eletr\'onica, Telecomunica\c c\~oes e Inform\'atica da Universidade de Aveiro.}
\EndTitlePage
\titlepage\ \endtitlepage % empty page


\TitlePage
  \vspace*{55mm}
  \TEXT{\textbf{o j\'uri\newline}}
       {}
  \TEXT{presidente}
       {\textbf{Professor Doutor Arnaldo Silva Rodrigues de Oliveira}\newline {\small
        Professor auxiliar do Departamento de Eletrónica, Telecomunicações e Informática da Universidade de Aveiro}}
  \vspace*{5mm}
  \TEXT{vogais}
       {\textbf{Professor Doutor Rui Pedro de Magalhães Claro Prior}\newline {\small
        Professor auxiliar da Faculdade de Ciências da Universidade do Porto}}
  \vspace*{5mm}
  \TEXT{}
       {\textbf{Professor Doutor Ilídio Fernando de Castro Oliveira}\newline {\small
        Professor auxiliar do Departamento de Eletrónica, Telecomunicações e Informática da Universidade de Aveiro}}
\EndTitlePage
\titlepage\ \endtitlepage % empty page

\TitlePage
  \vspace*{55mm}
  \TEXT{\textbf{agradecimentos}}
       {Agradeço à minha família, amigos e companheiros o apoio e incentivo necessário para que terminasse esta etapa tão importante.}
\EndTitlePage
\titlepage\ \endtitlepage % empty page
%A disponibilidade de dispositivos móveis e a pressão nos sistemas de saúde levaram a uma explosão do número de aplicações móveis para a saúde (\textit{mHealth}). Estas aplicações são cada vez mais relevantes para monitorização e acompanhamento de doentes, especialmente com doenças crónicas, mas, quando se trata de escolher arquiteturas de sistema para o seu desenvolvimento, não há respostas óbvias. Neste trabalho, estudamos algumas plataformas de \textit{backend} que podem ser utilizados por uma aplicação \textit{mHealth} e as respetivas implicações na arquitetura de sistema. 

%Como resultado, demonstramos o desenvolvimento de um sistema para um projeto de Investigação e Desenvolvimento que gere a informação fisiológica de pessoas através de uma aplicação móvel. Estas pessoas são então participantes de um estudo, podendo estas ser monitorizadas remotamente por revisores através de uma aplicação web. Para guardar todos os dados adaptamos uma plataforma já desenvolvida para suportar aplicações de \textit{mHealth} que foi uma das exploradas no fase de estudo e exploração desta dissertação.
\TitlePage
  \vspace*{55mm}
   \TEXT{\textbf{Palavras Chave}}{\textit{mHealth}, Computação Móvel, Plataformas de \textit{backend}, Monitorização fisiológica}
  \vspace*{10mm}
  \TEXT{\textbf{Resumo}}
       {Os dispositivos móveis são cada vez mais utilizados na recolha de dados fisiológicos, quer para uso pessoal ou quer para suporte à prestação de cuidados de saúde. Esta área é mais conhecida como \textit{mHealth}. O programador de aplicações \textit{mHealth} precisa de desenvolver o \textit{front-end} móvel, para os utilizadores finais e o \textit{backend}, para a integração de sistemas e persistência dos dados recolhidos. Existem várias possibilidades de escolha relativamente à arquitetura a utilizar para o \textit{backend}, que vão desde tecnologias empresariais genéricas a tecnologias específicas de determinados domínios da área de saúde. Neste trabalho, estudamos três plataformas para o \textit{backend} de aplicações \textit{mHealth (Open mHealth, HL7 FHIR, Google Fit)}. Foram realizadas implementações exploratórias e aprendidas lições.}
    \TEXT{}
       {De seguida, mostramos a utilização da plataforma \textit{Open mHealth} numa aplicação mais abrangente, cobrindo casos típicos da utilização de \textit{mHealth} num cenário de investigação: criação e gestão de estudos, monitorização ambulatória de participantes e análise dos dados por especialistas em saúde. A arquitetura proposta foi implementada num protótipo funcional, permitindo a monitorização de frequência cardíaca, eletrocardiograma e acelerómetro através de smartphones \textit{Android}, e a revisão dos casos por especialistas numa aplicação \textit{web} dedicada.   
       }
\EndTitlePage
\titlepage\ \endtitlepage % empty page

\TitlePage
  \vspace*{55mm}
  \TEXT{\textbf{Keywords}}{mHealth, Mobile computing, Backend platforms, Physiologic monitoring}
  \vspace*{10mm}
  \TEXT{\textbf{Abstract}}
       { Mobile devices are increasingly being used in the collection of physiological data, for personal use or health care support. This field of application is known as mHealth. The developer of mHealth applications need to address the mobile front-end, for the final users, and the backend, for systems integration and long-term persistence. Several architectural choices can be adopted for the backend, ranging from general purpose enterprise-level technologies, to health-domain specific frameworks. In this work, we study selected candidate frameworks for the backend of mHealth applications (Open mHealth, HL7 FHIR, Google Fit). Exploratory implementations were conducted and lessons learned.}
   \TEXT{}{We then show the use of the Open mHealth framework in a more comprehensive application, covering typical mHealth use cases in a research scenario: studies creation and management, ambulatory monitoring of participants and data analysis by health experts. The proposed architecture was implemented in a functional prototype, allowing for heart rate, electrocardiogram, accelerometer and activity monitoring with Android smartphones, and cases review by experts in a dedicated web application. }
\EndTitlePage
\titlepage\ \endtitlepage % empty page


%
% Tables of contents, of figures, ...
%

\pagenumbering{roman}

\tableofcontents

\cleardoublepage
\listoffigures

\cleardoublepage
\listoftables

\cleardoublepage


\printglossary[style=mcolindex,title=Lista de Abrevia\c c\~oes e Acr\'onimos, nonumberlist]


\setacronymstyle{long-short}

\newacronym{oms}{OMS}{Organiza\c c\~ao Mundial de Sa\'ude}
\newacronym{mHealth}{mHealth}{Mobile Health}
\newacronym{ECG}{ECG}{Eletrocardiograma}
\newacronym{API}{API}{Application Programming Interface}
\newacronym{XML}{XML}{eXtensible Markup Language}
\newacronym{HTML}{HTML}{HyperText Markup Language}
\newacronym{HTTP}{HTTP}{Hypertext Transfer Protocol}
\newacronym{JSON}{JSON}{JavaScript Object Notation}
\newacronym{REST}{REST}{Representational State Transfer}
\newacronym{URI}{URI}{Uniform Resource Identifier}
\newacronym{CRUD}{CRUD}{Create, Read, Update and Delete}
\newacronym{OMH}{OMH}{Open mHealth}
\newacronym{HL7}{HL7}{Health Level Seven}
\newacronym{FHIR}{FHIR}{Fast Healthcare Interoperability Resources}
\newacronym{RFC}{RFC}{Request For Comments}
\newacronym{JVM}{JVM}{Java Virtual Machine}
\newacronym{RIM}{RIM}{Reference Information Model}
\newacronym{ID}{I\&D}{Investigação e Desenvolvimento}
\newacronym{DSU}{DSU}{Data Storage Unit}
\newacronym{SIG}{SIG}{Special Interest Group}
\newacronym{HDP}{HDP}{Health Device Profile}
\newacronym{SPP}{SPP}{Serial Port Profile}
\newacronym{SDK}{SDK}{Software Development Kit}
\newacronym{CSS}{CSS}{Cascading Style Sheets}
\newacronym{IEETA}{IEETA}{Instituto de Engenharia Eletrónica e Telemática de Aveiro}
% The chapters (usually written using the isolatin font encoding ...)

\cleardoublepage
\pagenumbering{arabic}


\import{chapters/}{chapter1.tex}

\import{chapters/}{chapter2.tex}

\import{chapters/}{chapter3.tex}

\import{chapters/}{chapter4.tex}

\import{chapters/}{chapter5.tex}

\import{chapters/}{chapter6.tex}

\import{chapters/}{chapter7.tex}

\import{chapters/}{chapter8.tex}


\renewcommand{\bibname}{Referências}
\bibliography{references}

\cleardoublepage

\end{document}
 